\documentclass[a4paper,10pt]{jsarticle}

% レイアウト
\setlength{\textwidth}{\fullwidth}
\setlength{\textheight}{39\baselineskip}
\addtolength{\textheight}{\topskip}
\setlength{\voffset}{-0.5in}
\setlength{\headsep}{0.3in}
\pagestyle{myheadings}

% パッケージ
\usepackage[dvipdfmx]{graphicx}
\usepackage{amsmath,amssymb,epsfig}
\usepackage{bm}
\usepackage{ascmac}
\usepackage{pifont}
\usepackage{multirow}
\usepackage{enumerate}
\usepackage{cases}
\usepackage{type1cm}
\usepackage{cancel}
\usepackage{url}
\usepackage{color}
\usepackage{listings,jlisting}
% 大きな中括弧
\usepackage{cases}

% 定義
\DeclareMathOperator*{\argmin}{arg\,min}
\DeclareMathOperator*{\argmax}{arg\,max}
\def\vec#1{\mbox{\boldmath$#1$}}
\def\R{{\Bbb R}}

% カウンタの設定
\setcounter{section}{0}
\setcounter{subsection}{0}
\setcounter{subsubsection}{0}
\setcounter{equation}{0}

% キャプションの図をFigに変更
\renewcommand{\figurename}{Fig.}
\renewcommand{\tablename}{Tab.}

% 式番号を式(章番号.番号)に
% \makeatletter
% \renewcommand{\theequation}{\arabic{section}.\arabic{equation}}
% \@addtoreset{equation}{section}
% \makeatother

% プログラムに色をつける
\usepackage{color}

\definecolor{codegreen}{rgb}{0,0.6,0}
\definecolor{codegray}{rgb}{0.5,0.5,0.5}
\definecolor{codepurple}{rgb}{0.58,0,0.82}
\definecolor{backcolour}{rgb}{0.95,0.95,0.92}

\lstdefinestyle{mystyle}{
    backgroundcolor=\color{backcolour},
    commentstyle=\color{codegreen},
    keywordstyle=\color{magenta},
    numberstyle=\tiny\color{codegray},
    stringstyle=\color{codepurple},
    basicstyle=\footnotesize,
    breakatwhitespace=false,
    breaklines=true,
    captionpos=b,
    keepspaces=true,
    numbers=left,
    numbersep=5pt,
    showspaces=false,
    showstringspaces=false,
    showtabs=false,
    tabsize=2
}

\lstset{style=mystyle}

% 表紙
\title{知能システム学特論レポート}
\author{
(DL2班)Caffe on Ubuntu\\
}
\date{2015年\ 7月\ 21日}

% ドキュメントの開始
\begin{document}
\maketitle
\section{報告者}
\begin{list}{}{}
 \item 15344203\hspace{0.5cm} 有田 裕太
 \item 15344206\hspace{0.5cm} 緒形 裕太
 \item 15344209\hspace{0.5cm} 株丹 亮
 \item 12104125\hspace{0.5cm} 宮本 和
\end{list}

\section{進行状況}

\begin{itemize}
\item 畳み込みネットワークと正規化層の理論について
\item 独自の訓練データにより作成した識別器を用いた識別結果
\end{itemize}

\section{理論研究}

%%%%%% ogata %%%%%%
\subsection{誤差関数}
本班のcaffeを用いた画像分類では,第4回レポートで記したような多クラス分
類を行っており,出力関数はソフトマックス関数である.多クラス分類ではネッ
トワークが実現する関数を各クラスの事後確率のモデルであると見なし,そのモ
デルのもとで訓練データに対するネットワークパラメータの尤度を評価し,これ
を最大化する.いま訓練データとして,入力${\bf x}$とその正解クラスの組$C_k$が与え
られたとする.このときの目標出力の
2値の値を$K$個並べたベクトル${\bf
d}_n$によって表現すると,事後分布は次式のようになる.

\begin{equation}
 p({\bf d}|{\bf x}) = \prod_{k=1}^{K}p(C_k|{\bf x})^{d_k}
\end{equation}

これより,訓練データ${({\bf x}_n|{\bf d}_n)}(n=1,...,N)$に対する{\bf w}
の尤度は

\begin{equation}
 L({\bf w})=\prod_{n=1}^{N}p({\bf x}_n|{\bf x}_n;{\bf
	w})=\prod_{n=1}^{N}\prod_{k=1}^{K}p(C_k|{\bf
	x})^{d_{nk}}=\prod_{n=1}^{N}\prod_{k=1}^{K}(y_k({\bf x};{\bf w}))^{d_{nk}}
\end{equation}

と導ける.この尤度の対数とって符号を反転した次の式を誤差関数として用いる.
この関数は交差エントロピー(cross entropy)と呼ばれる.
\begin{eqnarray}
 E({\bf w})=-\sum^{N}_{n=1}\sum^{N}_{k=1}d_{nk}\log
	y_{k}(\vec{x}_{n};{\bf w})
\end{eqnarray}


\section{プログラミング}

\section{今後の課題}
\begin{itemize}
 \item 理論研究を進める.
 \item データセットの作成,学習実行結果の評価と過程の可視化.
\end{itemize}

\end{document}