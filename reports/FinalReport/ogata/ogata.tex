\documentclass[a4paper,10pt]{jsarticle}

% レイアウト
\setlength{\textwidth}{\fullwidth}
\setlength{\textheight}{39\baselineskip}
\addtolength{\textheight}{\topskip}
\setlength{\voffset}{-0.5in}
\setlength{\headsep}{0.3in}
\pagestyle{myheadings}

% パッケージ
\usepackage[dvipdfmx]{graphicx}
% \usepackage[hypertex]{hyperref}
\usepackage{amsmath,amssymb,epsfig}
\usepackage{bm}
\usepackage{ascmac}
\usepackage{pifont}
\usepackage{multirow}
\usepackage{enumerate}
\usepackage{cases}
\usepackage{type1cm}
\usepackage{cancel}
\usepackage{url}
\usepackage[dvipdfmx]{color}
\usepackage{listings,jlisting}
\usepackage{here}
% 大きな中括弧
\usepackage{cases}

% 定義
\DeclareMathOperator*{\argmin}{arg\,min}
\DeclareMathOperator*{\argmax}{arg\,max}
\def\vec#1{\mbox{\boldmath$#1$}}
\def\R{{\Bbb R}}

% カウンタの設定
\setcounter{section}{0}
\setcounter{subsection}{0}
\setcounter{subsubsection}{0}
\setcounter{equation}{0}

% キャプションの図をFigに変更
\renewcommand{\figurename}{Fig.}
\renewcommand{\tablename}{Tab.}

% 式番号を式(章番号.番号)に
% \makeatletter
% \renewcommand{\theequation}{\arabic{section}.\arabic{equation}}
% \@addtoreset{equation}{section}
% \makeatother

% プログラムに色をつける
\usepackage{color}

\definecolor{codegreen}{rgb}{0,0.6,0}
\definecolor{codegray}{rgb}{0.5,0.5,0.5}
\definecolor{codepurple}{rgb}{0.58,0,0.82}
\definecolor{backcolour}{rgb}{0.95,0.95,0.92}

\lstdefinestyle{mystyle}{
    backgroundcolor=\color{backcolour},
    commentstyle=\color{codegreen},
    keywordstyle=\color{magenta},
    numberstyle=\tiny\color{codegray},
    stringstyle=\color{codepurple},
    basicstyle=\footnotesize,
    breakatwhitespace=false,
    breaklines=true,
    captionpos=b,
    keepspaces=true,
    numbers=left,
    numbersep=5pt,
    showspaces=false,
    showstringspaces=false,
    showtabs=false,
    tabsize=2
}

\lstset{style=mystyle}

% 表紙
\title{知能システム学特論\ 最終レポート}
\author{
DL2班\hspace{5mm}15344206 \hspace{5mm}緒形 裕太\\
}
\date{2015年\ 8月\ 27日}

% ドキュメントの開始
\begin{document}
\maketitle

\section{テーマ}
\label{sec:テーマ}
ubuntu 上での Caffe による画像認識,多クラス分類.
% section テーマ (end)

\section{概要}
\label{sec:概要}
Deep Learning ,深層学習と呼ばれる多層ニューラルネットワークの一種を用いて入力した画像の認識,多
クラス分類を行った. Deep Learning の実装には Caffe と呼ばれる開発フレームワークを用いた. Caffe は
GPU にも対応し,処理が高速,学習済のモデルも提供しているといった特徴がある.
% section 概要 (end)

\section{自分の担当範囲}
\label{sec:自分の担当範囲}
主に, Caffe で用いられている方法である畳み込みニューラルネットの理論研究を担当した.また画像認識
の過程で出力される中間層の可視化,考察なども行った.
% section 自分の担当範囲 (end)

\section{感想}
\label{sec:感想}
最近良く耳にする Deep Learning について,単語はよく聞くが実際どのようなもので,どのような仕組み
で動いているのかは知らなかったが,この授業を通して詳細に学習することができた.また Caffe のインス
トールや中間層の出力など, Deep Learning には直接関係ないが,システムを用いて何かを動かすというスキ
ルが磨かれたと思うので,今後別の場面でも役立てたい.実際に自分の手を動かしてみないと何事も学べない
と感じた.
% section 感想 (end)

\section{評価}
\label{sec:評価}
\begin{table}[H]
  \begin{center}
    \begin{tabular}{l|p{12cm}} \hline
      名前 & 評価 \\ \hline \hline
      自分 & 理論研究を主に行った.しかしプログラミングの面ではほとんど活躍できなかった
ので,またこのよう機会があるときには役に立てるように努力したい.有田君,株
丹君,宮本さんに非常に感謝している. \\ \hline
      有田 裕太 & プログラミングの面で非常に頑張ってくれた.主にアニメキャラクターの識別を
行って,良い結果を出してくれた. \\ \hline
      宮本 和 & 理論研究を主に頑張ってくれた.参考文献を熟読して分かりやすい説明してくれ
た. \\ \hline
      株丹 亮 & 有田君と同じく,プログラミングの面で活躍してくれた. Caffe のインストール方
法を詳細に示してくれた.また実在のアイドルの識別を行い,結果を考察して研究
に厚みを持たせてくれた.\\ \hline
    \end{tabular}
  \end{center}
\end{table}
% section 評価 (end)


\begin{thebibliography}{99}
  \addcontentsline{toc}{section}{参考文献}
 \bibitem{okatani} 岡谷貴之,``機械学習プロフェッショナルシリーズ 深層学習'',講談社,2015.
 \bibitem{SIG2D} SIG2D,``SIG2D' 14 Proceedings of the 3rd Interdimensional Conference on 2D Information Processing '',\url{http://sig2d.org/publications/},2014.
 \end{thebibliography}
\end{document}