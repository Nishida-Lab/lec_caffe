\documentclass[a4paper,10pt]{jsarticle}

% レイアウト
\setlength{\textwidth}{\fullwidth}
\setlength{\textheight}{39\baselineskip}
\addtolength{\textheight}{\topskip}
\setlength{\voffset}{-0.5in}
\setlength{\headsep}{0.3in}
\pagestyle{myheadings}

% パッケージ
\usepackage[dvipdfmx]{graphicx}
\usepackage{amsmath,amssymb,epsfig}
\usepackage{bm}
\usepackage{ascmac}
\usepackage{pifont}
\usepackage{multirow}
\usepackage{enumerate}
\usepackage{cases}
\usepackage{type1cm}
\usepackage{cancel}
\usepackage{url}
\usepackage{listings,jlisting}
% 大きな中括弧
\usepackage{cases}

% 定義
\DeclareMathOperator*{\argmin}{arg\,min}
\DeclareMathOperator*{\argmax}{arg\,max}
\def\vec#1{\mbox{\boldmath$#1$}}
\def\R{{\Bbb R}}

% カウンタの設定
\setcounter{section}{0}
\setcounter{subsection}{0}
\setcounter{subsubsection}{0}
\setcounter{equation}{0}

% キャプションの図をFigに変更
\renewcommand{\figurename}{Fig.}
\renewcommand{\tablename}{Tab.}

% 式番号を式(章番号.番号)に
\makeatletter
\renewcommand{\theequation}{\arabic{section}.\arabic{equation}}
\@addtoreset{equation}{section}
\makeatother

% 表紙
\title{知能システム学特論レポート}
\author{
(DL2班)Caffe on Ubuntu\\
}
\date{2015年\ 7月\ 6日}

% ドキュメントの開始
\begin{document}
\maketitle
\section{報告者}
\begin{list}{}{}
 \item 15344203\hspace{0.5cm} 有田 裕太
 \item 15344206\hspace{0.5cm} 緒形 裕太
 \item 15344209\hspace{0.5cm} 株丹 亮
 \item 12104125\hspace{0.5cm} 宮本 和
\end{list}

\section{進行状況}

\begin{itemize}
\item 理論研究
\item 畳み込みネットワークについて
\end{itemize}

\section{理論研究}

\subsection{畳込み層}

実用的な畳込みネットでは,グレースケールの画像1枚に対してではなく,多チャネルの画像に対し,複数個のフィルタを並行して畳込む演算を行う.チャネルの画像とは各画素が複数の値を持つ画像であり,チャネル数がKの画像の各画素はK個の値を持つ.例えば,グレースケールの画像では K = 1,RGBの3色からなるカラー画像では K = 3 となる.畳込みネットの中間層では,さらにそれ以上のチャネル数の画像を扱う.以下では,画像の縦横の画素数が W × W でチャネル数が K のとき,画像のサイズを W × W × K と書く.

濃淡地を各画素に格納したグレースケールの画像を考える.
画像サイズを$W\times W$画素とし,画素をインデックス$(i,j)(i = 0,\cdots,W-1, j = 0,\cdots,W-1)$で表す.
これまでインデックスは1から開始していたが,画像(及びそれに類するもの)を扱う場合はインデックスを0から始めることとする.
画素$(i,j)$の画素値を$x_{ij}$と書き,負の値を含む実数値をとるとする.
そして,この画像に適用する $H\times H$画素のフィルタ
(サイズの小さい画像)を考える.このフィルタのインデックス$(p,q)(p=0,\cdots,H-1, q=0,\cdots,H-1)$で表し,画素値を$h_{pq}$と書く.

\begin{figure}[t]
 \centering
 \includegraphics[scale=0.4]{fig/eps/dl67.eps}
  \caption{畳み込み層の概要}
  \label{fig:畳み込み層の概要}
\end{figure}

図1を用いて畳込み層での計算を説明する.この畳込み層は直前の層からKチャネルの画像 $x_{ijk} (k = 0,...,K − 1)$ を受け取り,これに M = 3 種類のフィルタ $h_{pqkm} (m = 0,...,M − 1)$ を適用している.各フィルタ (m = 0, 1, 2) は通常,入力と同じチャネル数Kを持ち(サイズを H × H × K とする),図1のようにフィルタごとに計算は並行に実行される.計算の中身は,そのフィルタの各チャネルごとに,これも並行に画像とフィルタの畳込みを行った後,結果を画素ごとに全チャネルにわたって加算する.

\begin{equation}
  u_{ijm} = \sum_{p=0}^{K-1} \sum_{q=0}^{H-1} \sum_{q=0}^{H-1} z_{i+p,j+q,k}^{(l-1)} h_{pqkm}+b_{ijm}
\end{equation}

入力画像のチャネル数によらず,1つのフィルタからの出力は常に1チャネルになる.

%%%%%%%%%%% ogata %%%%%%%%%%%
次にこうして得た$u_{ijm}$に活性化関数を適応する.
\begin{equation}
 z_{ijm}=f(u_{ijm})
\end{equation}
図\ref{fig:畳み込み層の概要}のように,この値が畳み込み層の最終的な出力と
なり,その後の層へと伝わる.これらはフィルタ数$M$と同数のチャネル数を持
つ多チャネルの画像とみなせる.つまり入力サイズが$W\times W\times K$のとき,出力サ
イズは$W\times W\times M$になる.

式(6.2),(6.3)は,層間に特別な構造の単層ネットワークとして表現できる.こ
の層の入力と出力のユニット数はそれぞれ$W\times W\times K$および$W\times W\times M$であり,畳
み込み計算の局所性を反映して,出力層のユニットとのみ結合する.その結合の
重みがフィルタ係数$h_{pqkm}$である.この重みは出力層の同一チャネルの全ユ
ニットで共有される.これは重み共有(weigh sharing, weight tying)と呼ばれ,
このような結合の局所性と重みを共有することが畳み込み層の特徴である.

\subsection{プーリング}
プーリング層は通常,畳み込み層の直後に設置される.プーリング層は畳み込み層で抽出された特徴の位置感度を低下させる働きがあり,対象とする特徴量の画像内での位置が若干変化した場合においても,プーリング層の出力が不変になるようにする.

プーリング層での計算は次のようにして行う.サイズ$W \times W \times K$の入力画像上で画素$(i,j)$を中心とする$H\times H$の正方領域とり,この中に含まれる画素の集合を$P_{ij}$で表す.$P_{ij}$内の画素についてチャネル$k$ごとに独立に,$H^2$個ある画素値を使って1つの画素値$u_{ijk}$を求める.これにはいくつかの方法があり,最大プーリング(max pooling)は,$H^{2}$個の画素値の最大値を選ぶ.
\begin{eqnarray}
 u_{ijk} = \max_{(p,q) \in P_{ij}} z_{pqk}
\end{eqnarray}
また,平均プーリング(average pooling)はそれらの平均値を計算する.
\begin{eqnarray}
 u_{ijk} = \frac{1}{H^{2}}\sum_{(p,q) \in P_{ij}} z_{pq}
\end{eqnarray}
通常は,プーリング層の出力のチャネル数は入力画像のチャネル数と一致する.
これらのプーリングを含む一般性を持った表記として,次のLpプーリング(Lp pooling)があります.
\begin{eqnarray}
 u_{ijk} = \left(\frac{1}{H^{2}}\sum_{(p,q) \in P_{ij}}z^{P}_{pqk}\right)^{\frac{1}{P}}
\end{eqnarray}
$P=1$で平均プーリング,$P=\inf$で最大プーリングが表現できる.なお,画像認識では最大プーリングが良く用いられている.

プーリング層では2以上のストライドを設定することが普通である.プーリング層では結合の重みは固定されており,学習によって変化するパラメータはない.


\section{プログラミング}
\subsection{学習を実行するPC環境の見直し}
今までcaffeのサンプル実行,中間層の出力はCPUのみの演算で十分な応答を得られた.
しかし与えられたデータセットにおいて学習を行うプログラムを実行する場合,非常に時間がかかることがわかった.

現時点では学習を行うために十分なデータ数を有するデータセットを準備できていないため,大規模なデータセットで学習済みの状態から目的とする別のデータセットへ学習し直す方法(ファインチューニング)を行った.
ファインチューニングは学習データがあまり多くない場合でも学習を行うことができる.
この手法を用いて試験的に学習用データとテスト用データを6枚ずつ用意して学習を行ったが,ファインチューニングが完全に完了するまで約3時間かかった.

このことからもこれからの課題研究を可能な限り円滑に行うために,学習時にはGPUを使った並列計算が必要であることがわかった.
したがってcaffeがサポートしているNVIDIAの「CUDA」及び,Deep Learning用のCUDAライブラリ「cuDNN」を使用する必要があると考える.
後者のライブラリはデベロッパー登録申請(CUDA Registered Developer Program)が必要であるので申請を行い,認可を待っている状態である.

\section{今後の課題}
\begin{itemize}
 \item 理論研究を進める.
 \item Caffeを使いこなす
\end{itemize}

\end{document}