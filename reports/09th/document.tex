\documentclass[a4paper,10pt]{jsarticle}

% レイアウト
\setlength{\textwidth}{\fullwidth}
\setlength{\textheight}{39\baselineskip}
\addtolength{\textheight}{\topskip}
\setlength{\voffset}{-0.5in}
\setlength{\headsep}{0.3in}
\pagestyle{myheadings}

% パッケージ
\usepackage[dvipdfmx]{graphicx}
\usepackage{amsmath,amssymb,epsfig}
\usepackage{bm}
\usepackage{ascmac}
\usepackage{pifont}
\usepackage{multirow}
\usepackage{enumerate}
\usepackage{cases}
\usepackage{type1cm}
\usepackage{cancel}
\usepackage{url}
\usepackage{color}
\usepackage{listings,jlisting}
% 大きな中括弧
\usepackage{cases}

% 定義
\DeclareMathOperator*{\argmin}{arg\,min}
\DeclareMathOperator*{\argmax}{arg\,max}
\def\vec#1{\mbox{\boldmath$#1$}}
\def\R{{\Bbb R}}

% カウンタの設定
\setcounter{section}{0}
\setcounter{subsection}{0}
\setcounter{subsubsection}{0}
\setcounter{equation}{0}

% キャプションの図をFigに変更
\renewcommand{\figurename}{Fig.}
\renewcommand{\tablename}{Tab.}

% 式番号を式(章番号.番号)に
% \makeatletter
% \renewcommand{\theequation}{\arabic{section}.\arabic{equation}}
% \@addtoreset{equation}{section}
% \makeatother

% プログラムに色をつける
\usepackage{color}

\definecolor{codegreen}{rgb}{0,0.6,0}
\definecolor{codegray}{rgb}{0.5,0.5,0.5}
\definecolor{codepurple}{rgb}{0.58,0,0.82}
\definecolor{backcolour}{rgb}{0.95,0.95,0.92}

\lstdefinestyle{mystyle}{
    backgroundcolor=\color{backcolour},
    commentstyle=\color{codegreen},
    keywordstyle=\color{magenta},
    numberstyle=\tiny\color{codegray},
    stringstyle=\color{codepurple},
    basicstyle=\footnotesize,
    breakatwhitespace=false,
    breaklines=true,
    captionpos=b,
    keepspaces=true,
    numbers=left,
    numbersep=5pt,
    showspaces=false,
    showstringspaces=false,
    showtabs=false,
    tabsize=2
}

\lstset{style=mystyle}

% 表紙
\title{知能システム学特論レポート}
\author{
(DL2班)Caffe on Ubuntu\\
}
\date{2015年\ 7月\ 16日}

% ドキュメントの開始
\begin{document}
\maketitle
\section{報告者}
\begin{list}{}{}
 \item 15344203\hspace{0.5cm} 有田 裕太
 \item 15344206\hspace{0.5cm} 緒形 裕太
 \item 15344209\hspace{0.5cm} 株丹 亮
 \item 12104125\hspace{0.5cm} 宮本 和
\end{list}

\section{進行状況}

\begin{itemize}
\item 畳み込みネットワークと正規化層の理論について
\item データセットの作成準備
\end{itemize}

\section{理論研究}
\subsection{勾配の計算}


\subsection{単一チャネルの正規化}
単一チャネルの画像$x_{ij}$に対し,プーリングと同様,画素$(i,j)$を中心と
する$H\times H$の正方領域$P_{ij}$を考える.減算正規化とは,入力画像の各
画素濃淡から,$P_{ij}$に含まれる画素の濃淡の平均,つまり
$\bar{x}_{ij}= \sum_{(p,q)\in{P_{ij}}}^{} x_{i+p,j+q}$を差し引く.
\begin{equation}
 z_{ij} = x_{ij}-\bar{x}_{ij}
\end{equation}
ここで差し引く$\bar{x}_{ij}$には,重み付き平均
\begin{equation}
 \bar{x}_{ij}=\sum_{(p,q)\in{P_{ij}}} w_{pq}x_{i+p,j+q}
\end{equation}
を使う場合もある.その場合$w_{pq}$は
\begin{equation}
  \sum_{(p,q)\in{P_{ij}}}^{} w_{pq} = \sum_{q=0}^{H-1} \sum_{q=0}^{H-1} w_{pq}=1
\end{equation}
であり,領域の中央で最大値をとり,周辺部へ向けて低下するようなものとする.
領域の中央部をより重視し,周辺部の相対的な影響を少なくるためである.

同じ領域内で,さらに画素値の分散を抑える操作が除算正規化である.$P_{ij}$
内の画素値の分散は
\begin{equation}
 \sigma^2_{ij}=\sum_{(p,q)\in{P_{ij}}}^{} w_{pq}(x_{i+p,j+q}-\bar{x}_{ij})^2
\end{equation}
となるが,減算正規化を施した入力画像をこの標準偏差で割る.
\begin{equation}
 z_{ij}=\frac{x_{ij}-\bar{x}_{ij}}{\sigma_{ij}}
\end{equation}
この計算をそのまま行うと,濃淡変化が少ない局所領域ほど濃淡変化が増幅され,
ノイズが強調される.そこで,入力画像のコントラストが大きい部分にのみ適用
するために,ある定数$c$を設定し,濃淡の標準偏差がこれを下回る
($\sigma_{ij}<c$)で除算する
\begin{equation}
 z_{ij}=\frac{x_{ij}-\bar{x}_{ij}}{max(\sigma_{ij}<c)}
\end{equation}
や,同様の効果が$\sigma_{ij}$に応じて連続的に変化する
\begin{equation}
 z_{ij}=\frac{x_{ij}-\bar{x}_{ij}}{\sqrt{\sigma_{ij}+c}}
\end{equation}
を使う.% 減算正規化および式\ref{}による除算正規化の計算例を図
% \ref{}に示す.
% これらの正規化の結果,画像の画素値は負の値をとり得るため,画素値の最大値
% と最小値が[0,255]の範囲に収まるように画素値を線形変換している.

\subsection{多チャネル画像の正規化}
多チャネル画像では,単一チャネルごとの正規化を適応することもできるが,一
般的にはチャネル間の相互作用を考える.その場合,画素値の平均と分散を求め
る対象が全チャネルにわたる局所領域$P_{ij}$内の画素の集合に代わる.$K$チャ
ネルからなる画像$x_{ijk}$を対象とするとき,重み付き平均を
\begin{equation}
 \bar{x}_{ij}=\frac{1}{K}\sum_{k=0}^{K-1}\sum_{(p,q)\in{P_{ij}}}^{} w_{pq}x_{i+p,j+q,k}
\end{equation}
のように決定します.減算正規化は,画素$(i,j)$ごとに違うがチャネル間では
共通の$\bar{x}_ij$を差し引いて
\begin{equation}
 z_{ijk}=x_{ijk}-\bar{x}_{ij}
\end{equation}
のように行われる.画像の全チャネルにわたる$P_{ij}$の分散は
\begin{equation}
  \omega^2_{ij}=\frac{1}{K}\sum_{k=0}^{K-1}\sum_{(p,q)\in{P_{ij}}}^{} w_{pqk}(x_{i+p,j+q,k}-\bar{x}_{ij})^2
\end{equation}
のように計算され,除算正規化は
\begin{equation}
  z_{ijk}=\frac{x_{ijk}-\bar{x}_{ij}}{max(\sigma_{ij},c)}
\end{equation}
となる.単一チャネルの場合と同様に,分母を$\sqrt{c+\sigma^2_{ij}}$とする
こともできる.


\section{プログラミング}
\subsection{学習パラメータの設定}
学習を行う上で必要なパラメータについて説明する.この設定が記述されているファイルはcifar10\_quick\_solver.prototxtである.
以下に設定ファイルの内容を示し,各パラメータに関する意味を記述する.

\begin{lstlisting}[basicstyle=\ttfamily\footnotesize, frame=single, firstnumber=1, numbers=left, breaklines=true]
net: "examples/cifar10/cifar10_quick_train_test.prototxt"
test_iter: 100
test_interval: 500
base_lr: 0.0001
momentum: 0.9
weight_decay: 0.004
lr_policy: "fixed"
display: 100
max_iter: 4000
snapshot: 4000
snapshot_prefix: "examples/cifar10/cifar10_quick"
solver_mode: GPU
\end{lstlisting}

ここでバッチ数と呼ばれる単位を導入する.バッチ数は教師データを一度にいくつ処理するか(バッチサイズ)を決定し,これを1[batch]とする.
Caffeでは繰り返し数をバッチ数で指定している.

\begin{description}
  \item[net :]学習用ネットワーク定義ファイルを指定する.
  \item[test\_iter :]学習中の正答率評価を1回行うのに使う評価セットのデータ数をバッチ数で指定.評価セットのデータ数とバッチサイズの除算を行い,この数値を設定することで正答率評価にすべての評価セットを用いることができる.
  \item[test\_interval :]テストデータから正答率評価を行う間隔をバッチ数で指定.データ数が多い場合,正答評価に多くの時間がかかるので用いるデータセットの規模によって適切な値に設定する必要がある.
  \item[base\_lr,\ momentum,\ weight\_decay,\ lr\_policy :]学習率に関する設定.
  \item[display :]学習中のステータスを出力する回数をバッチ数で指定.
  \item[max\_iter :]学習の計算を最大どれだけ続けるかを訓練データのバッチ数で指定.ここで指定された数値とバッチサイズの積算が学習が終了するまでの処理する画像枚数となる.
  \item[snapshot,\ snapshot\_prefix :]学習の途中経過を保存する間隔と場所を指定.
  \item[solver\_mode :]学習をCPUのみ,あるいはGPUを用いるかを指定.
\end{description}


\section{今後の課題}
\begin{itemize}
 \item 理論研究を進める.
 \item データセットの作成,学習実行結果の評価と過程の可視化.
\end{itemize}

\end{document}