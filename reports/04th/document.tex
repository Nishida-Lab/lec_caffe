\documentclass[a4paper,10pt]{jsarticle}

% レイアウト
\setlength{\textwidth}{\fullwidth}
\setlength{\textheight}{39\baselineskip}
\addtolength{\textheight}{\topskip}
\setlength{\voffset}{-0.5in}
\setlength{\headsep}{0.3in}
\pagestyle{myheadings}

% パッケージ
\usepackage[dvipdfmx]{graphicx}
\usepackage{amsmath,amssymb,epsfig}
\usepackage{bm}
\usepackage{ascmac}
\usepackage{pifont}
\usepackage{multirow}
\usepackage{enumerate}
\usepackage{cases}
\usepackage{type1cm}
\usepackage{cancel}
\usepackage{url}
\usepackage{listings,jlisting}
% 大きな中括弧
\usepackage{cases}

% 定義
\DeclareMathOperator*{\argmin}{arg\,min}
\DeclareMathOperator*{\argmax}{arg\,max}
\def\vec#1{\mbox{\boldmath$#1$}}
\def\R{{\Bbb R}}

% カウンタの設定
\setcounter{section}{0}
\setcounter{subsection}{0}
\setcounter{subsubsection}{0}
\setcounter{equation}{0}

% キャプションの図をFigに変更
\renewcommand{\figurename}{Fig.}
\renewcommand{\tablename}{Tab.}

% 式番号を式(章番号.番号)に
\makeatletter
\renewcommand{\theequation}{\arabic{section}.\arabic{equation}}
\@addtoreset{equation}{section}
\makeatother

% 表紙
\title{知能システム学特論レポート}
\author{
(DL2班)Caffe on Ubuntu\\
}
\date{2015年\ 6月\ 29日}

% ドキュメントの開始
\begin{document}
\maketitle
\section{報告者}
\begin{list}{}{}
 \item 15344203\hspace{0.5cm} 有田 裕太
 \item 15344206\hspace{0.5cm} 緒形 裕太
 \item 15344209\hspace{0.5cm} 株丹 亮
 \item 12104125\hspace{0.5cm} 宮本 和
\end{list}

\section{進行状況}

\begin{itemize}
\item 理論研究
\item 順伝播型ネットワークについて
\end{itemize}


\section{理論研究}
\subsection{ユニットの出力}

\subsection{活性化関数}

\subsection{多層ネットワーク}

\subsection{出力層の設計と誤差関数}
\subsubsection{学習の枠組み}
順伝播型ネットワークが表現する関数$\vec{y}(\vec{x};\vec{w})$をネットワークのパラメータ$\vec{w}$を変えることで変化させ,望みの関数を与えることを考える.
入力$\vec{x}$と望みの出力$\mathrm{d}$のペアを次のように与える.
\begin{eqnarray}
 \{(\vec{x}_{1},\mathrm{d}_{1}),(\vec{x}_{1},\mathrm{d}_{1}),...,(\vec{x}_{N},\mathrm{d}_{N})\}
\end{eqnarray}
これらのペア$(\vec{x},\mathrm{d})$1つ1つを訓練サンプル(training samples)といい,その集合を訓練データ(training data)という.
ネットワーク$\vec{w}$を調整することで訓練データの入出力ペアをできるだけ再現すること学習という.

この場合,ネットワークが表す関数と訓練データとの近さ$(\vec{y}(\vec{x}_{n};\vec{w}))$を誤差関数(error function)で定義する.誤差関数は問題の種別や活性化関数によって異なる.表に問題の種別ごとの活性化関数と誤差関数の一覧を示す.

\begin{table}[htb]
\centering
\caption{問題の種別ごとの活性化関数と誤差関数}
\begin{tabular}[bt]{|c|c|c|}\hline
 問題の種別& 出力層の活性化関数&誤差関数 \\ \hline \hline
 回帰&正接双曲線関数や恒等写像 & 二乗誤差 式\\
 二値分類& ロジスティック関数& 式\\ 
 多クラス分類& ソフトマックス関数& 交差エントロピー 式\\ \hline
\end{tabular}
\end{table}

\subsubsection{回帰}
回帰(regression)とは出力連続値をとる関数を対象に訓練データを良く再現する関数を求めることをいう.回帰では活性化関数に正接双曲線関数や恒等写像を用い,評価関数は次式が良く用いられる.
\begin{eqnarray}
 E(\vec{w}) = \displaystyle{\frac{1}{2}}\sum^{N}_{n=1}||\vec{d}_{n}-\vec{y}(\vec{x}_{n};\vec{w})||^{2}
\end{eqnarray}

\subsubsection{二値分類}
二値分類では入力$\vec{x}$に応じて2種類に区別する問題を考える.すなわち,$d\in\{0,1\}$とする.このとき,活性化関数はロジスティック関数$y=1/(1+\exp(-u))$とし,誤差関数は次式で与える.
\begin{eqnarray}
 E(\vec{w})=-\sum^{N}_{n=1}\left[d_{n}\log y(\vec{x}_{n};\vec{w} + (1-d_{n})\log\{1-y(\vec{x}_{n};\vec{w})\})\right]
\end{eqnarray}

\section{今後の課題}
\begin{itemize}
 \item 理論研究を進める.
 \item Caffeを使いこなす
\end{itemize}

\end{document}