\documentclass[a4paper,10pt]{jsarticle}

% レイアウト
\setlength{\textwidth}{\fullwidth}
\setlength{\textheight}{39\baselineskip}
\addtolength{\textheight}{\topskip}
\setlength{\voffset}{-0.5in}
\setlength{\headsep}{0.3in}
\pagestyle{myheadings}

% パッケージ
\usepackage[dvipdfmx]{graphicx}
\usepackage{amsmath,amssymb,epsfig}
\usepackage{bm}
\usepackage{ascmac}
\usepackage{pifont}
\usepackage{multirow}
\usepackage{enumerate}
\usepackage{cases}
\usepackage{type1cm}
\usepackage{cancel}
\usepackage{url}
\usepackage{listings,jlisting}
% 大きな中括弧
\usepackage{cases}

% 定義
\DeclareMathOperator*{\argmin}{arg\,min}
\DeclareMathOperator*{\argmax}{arg\,max}
\def\vec#1{\mbox{\boldmath$#1$}}
\def\R{{\Bbb R}}

% カウンタの設定
\setcounter{section}{0}
\setcounter{subsection}{0}
\setcounter{subsubsection}{0}
\setcounter{equation}{0}

% キャプションの図をFigに変更
\renewcommand{\figurename}{Fig.}
\renewcommand{\tablename}{Tab.}

% 式番号を式(章番号.番号)に
\makeatletter
\renewcommand{\theequation}{\arabic{section}.\arabic{equation}}
\@addtoreset{equation}{section}
\makeatother

% 表紙
\title{知能システム学特論レポート}
\author{
(DL2班)Caffe on Ubuntu\\
}
\date{2015年\ 6月\ 29日}

% ドキュメントの開始
\begin{document}
\maketitle
\section{報告者}
\begin{list}{}{}
 \item 15344203\hspace{0.5cm} 有田 裕太
 \item 15344206\hspace{0.5cm} 緒形 裕太
 \item 15344209\hspace{0.5cm} 株丹 亮
 \item 12104125\hspace{0.5cm} 宮本 和
\end{list}

\section{進行状況}

\begin{itemize}
\item 理論研究
\item 順伝播型ネットワークについて
\end{itemize}


\section{理論研究}
\subsection{ユニットの出力}

\subsection{活性化関数}

\subsection{多層ネットワーク}

\subsection{出力層の設計と誤差関数}
\subsubsection{学習の枠組み}
順伝播型ネットワークが表現する関数$y$

\section{今後の課題}
\begin{itemize}
 \item 理論研究を進める.
 \item Caffeを使いこなす
\end{itemize}

\end{document}