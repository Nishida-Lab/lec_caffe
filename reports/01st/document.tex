\documentclass[a4paper,10pt]{jsarticle}

% レイアウト
\setlength{\textwidth}{\fullwidth}
\setlength{\textheight}{39\baselineskip}
\addtolength{\textheight}{\topskip}
\setlength{\voffset}{-0.5in}
\setlength{\headsep}{0.3in}
\pagestyle{myheadings}

% パッケージ
\usepackage[dvipdfmx]{graphicx}
\usepackage{amsmath,amssymb,epsfig}
\usepackage{bm}
\usepackage{ascmac}
\usepackage{pifont}
\usepackage{multirow}
\usepackage{enumerate}
\usepackage{cases}
\usepackage{type1cm}
\usepackage{cancel}
\usepackage{url}
\usepackage{listings,jlisting}
\usepackage[version=3]{mhchem}
% 大きな中括弧
\usepackage{cases}


% カウンタの設定
\setcounter{section}{0}
\setcounter{subsection}{0}
\setcounter{subsubsection}{0}
\setcounter{equation}{0}

% キャプションの図をFigに変更
\renewcommand{\figurename}{Fig.}
\renewcommand{\tablename}{Tab.}

% 式番号を式(章番号.番号)に
\makeatletter
\renewcommand{\theequation}{\arabic{section}.\arabic{equation}}
\@addtoreset{equation}{section}
\makeatother

% 表紙
\title{知能システム学特論レポート}
\author{
(DL2班)Caffe on Ubuntu\\
}
\date{平成27年\ 4月\ 30日}

% ドキュメントの開始
\begin{document}
% 表紙
\maketitle
\section{出席者:}
\begin{list}%
 {} %default label
 {} %formatting parameter
 \item 15344203 有田裕太
 \item 15344206 緒方裕太
 \item 15344209 株丹亮
\end{list}
\section{進行状況}
\subsection{理論調査}
DeeoLearningを使って何をするか.宇宙ゴミの検出

\subsection{プログラミング}
\begin{itemize}
 \item ソフトウェアのダウンロードとコンパイルはできた.
 \item サンプルの実行方法がまだわからない.
\end{itemize}

\begin{lstlisting}[basicstyle=\ttfamily\footnotesize, frame=single]
sudo apt-get install build-essential
\end{lstlisting}

\begin{lstlisting}[basicstyle=\ttfamily\footnotesize, frame=single]
sudo apt-get install -y libprotobuf-dev libleveldb-dev libsnappy-dev
libopencv-dev libboost-all-dev libhdf5-serial-dev protobuf-compiler gfortran libjpeg62
libfreeimage-dev libatlas-base-dev git python-dev python-pip
libgoogle-glog-dev libbz2-dev libxml2-dev libxslt-dev libffi-dev
libssl-dev libgflags-dev liblmdb-dev python-yaml
\end{lstlisting}

\begin{lstlisting}[basicstyle=\ttfamily\footnotesize, frame=single]
sudo easy_install pillow
\end{lstlisting}

\begin{lstlisting}[basicstyle=\ttfamily\footnotesize, frame=single]
git clone https://github.com/BVLC/caffe.git
cd caffe
\end{lstlisting}

\begin{lstlisting}[basicstyle=\ttfamily\footnotesize, frame=single]
cat python/requirements.txt | xargs -L 1 sudo pip install
\end{lstlisting}

\begin{lstlisting}[basicstyle=\ttfamily\footnotesize, frame=single]
sudo ln -s /usr/include/python2.7/ /usr/local/include/python2.7
sudo ln -s /usr/local/lib/python2.7/dist-packages/numpy/core/include/numpy/
/usr/local/include/python2.7/numpy
\end{lstlisting}

\begin{lstlisting}[basicstyle=\ttfamily\footnotesize, frame=single]
cp Makefile.config.example Makefile.config
gedit Makefile.config
\end{lstlisting}

\begin{lstlisting}[basicstyle=\ttfamily\footnotesize, frame=single]
## Refer to http://caffe.berkeleyvision.org/installation.html
# Contributions simplifying and improving our build system are welcome!

# cuDNN acceleration switch (uncomment to build with cuDNN).
# USE_CUDNN := 1

# CPU-only switch (uncomment to build without GPU support).
# CPU_ONLY := 1
\end{lstlisting}

\begin{lstlisting}[basicstyle=\ttfamily\footnotesize, frame=single]
CPU_ONLY := 1
\end{lstlisting}

\begin{lstlisting}[basicstyle=\ttfamily\footnotesize, frame=single]
# NOTE: this is required only if you will compile the python interface.
# We need to be able to find Python.h and numpy/arrayobject.h.
PYTHON_INCLUDE := /usr/include/python2.7 \
    /usr/lib/python2.7/dist-packages/numpy/core/include
\end{lstlisting}

\begin{lstlisting}[basicstyle=\ttfamily\footnotesize, frame=single]
make pycaffe
make all
make test
\end{lstlisting}

\begin{lstlisting}[basicstyle=\ttfamily\footnotesize, frame=single]
./scripts/download_model_binary.py models/bvlc_reference_caffenet
./data/ilsvrc12/get_ilsvrc_aux.sh
\end{lstlisting}

\begin{lstlisting}[basicstyle=\ttfamily\footnotesize, frame=single]
I0615 15:54:24.737364 17294 upgrade_proto.cpp:626] Successfully upgraded file specified
using deprecated V1LayerParameter
Traceback (most recent call last):
  File "python/classify.py", line 138, in <module>
    main(sys.argv)
  File "python/classify.py", line 110, in main
    channel_swap=channel_swap)
  File "/home/ry0/caffe/python/caffe/classifier.py", line 34, in __init__
    self.transformer.set_mean(in_, mean)
  File "/home/ry0/caffe/python/caffe/io.py", line 255, in set_mean
    raise ValueError('Mean shape incompatible with input shape.')
ValueError: Mean shape incompatible with input shape.
\end{lstlisting}

\begin{lstlisting}[basicstyle=\ttfamily\footnotesize, frame=single]
if ms != self.inputs[in_][1:]:
    raise ValueError('Mean shape incompatible with input shape.')
\end{lstlisting}

\begin{lstlisting}[basicstyle=\ttfamily\footnotesize, frame=single]
if ms != self.inputs[in_][1:]:
    print(self.inputs[in_])
    in_shape = self.inputs[in_][1:]
    m_min, m_max = mean.min(), mean.max()
    normal_mean = (mean - m_min) / (m_max - m_min)
    mean = resize_image(normal_mean.transpose((1,2,0)),in_shape[1:]).transpose((2,0,1))
    * (m_max - m_min) + m_min
\end{lstlisting}

\begin{lstlisting}[basicstyle=\ttfamily\footnotesize, frame=single]
import sys, numpy

categories = numpy.loadtxt(sys.argv[1], str, delimiter="\t")
scores = numpy.load(sys.argv[2])
top_k = 3
prediction = zip(scores[0].tolist(), categories)
prediction.sort(cmp=lambda x, y: cmp(x[0], y[0]), reverse=True)
for rank, (score, name) in enumerate(prediction[:top_k], start=1):
    print('#%d | %s | %4.1f%%' % (rank, name, score * 100))
\end{lstlisting}


\end{document}